\begin{abstract}
Overparameterization in deep learning typically refers to settings where a trained Neural Network (NN) has representational capacity to fit the training data in many ways, some of which generalize well, while others do not.
In the case of Recurrent Neural Networks (RNNs), there exists an additional layer of overparameterization, in the sense that a model may exhibit many solutions that generalize well for sequence lengths seen in training, some of which \emph{extrapolate} to longer sequences, while others do not.
Numerous works studied the tendency of Gradient Descent (GD) to fit overparameterized NNs with solutions that generalize well.
On the other hand, its tendency to fit overparameterized RNNs with solutions that extrapolate has been discovered only lately, and is far less understood.
In this paper, we analyze the extrapolation properties of GD when applied to overparameterized linear RNNs.
In contrast to recent arguments suggesting an implicit bias towards short-term memory, we provide theoretical evidence for learning low dimensional state spaces, which can also model long-term memory.
Our result relies on a dynamical characterization which shows that GD (with small step size and near-zero initialization) strives to maintain a certain form of balancedness, as well as on tools developed in the context of the \emph{moment problem} from statistics (recovery of a probability distribution from its moments).
Experiments corroborate our theory, demonstrating extrapolation via learning low dimensional state spaces with both linear and non-linear RNNs.
\end{abstract}