\section{Theoretical Analysis}\label{sec:analysis}

We turn to our theoretical analysis.
\Secref{sec:exact_extrapolation} proves that in the setting of Section~\ref{sec:setup}, convergence of GF to a zero loss solution leads the student to extrapolate, \emph{irrespective of how large its state space dimension is}.
\Secref{sec:approx_extrapolation} extends this result by establishing that, under mild conditions, approximate convergence leads to approximate extrapolation.
The results of sections \ref{sec:exact_extrapolation} and~\ref{sec:approx_extrapolation} assume that GF emanates from a balanced initialization, which empirically is known to capture near-zero initialization as commonly employed in practice (see \Secref{sec:setup}).
\Secref{sec:relaxed_balanced_assumption} theoretically supports this empirical premise, by showing that with high probability, random near-zero initialization leads to balancedness.



We introduce notations that will be used throughout the analysis. 
For a matrix $\mQ\in\R^{m\times n}$, we let $\|\mQ\|_F$, $\|\mQ\|_{\infty}$ and~$\|\mQ\|_2$ denote the Frobenius, $\ell_{\infty}$ and $\ell_2$ (spectral) norms, respectively.
For a vector $\vv \in \R^m$, we use $\|\vv \|$ to denote the Euclidean norm and~$\evv_i$ to denote its $i^{th}$ entry.

\subsection{Convergence Leads to Extrapolation} \label{sec:exact_extrapolation}

Theoretical analyses of implicit generalization often assume convergence to a solution attaining zero loss (see, e.g., \cite{azulay2021implicit, gunasekar2017implicit, lyu2019gradient, woodworth2020kernel}).
Under such an assumption, Theorem~\ref{thm:exact_extrapolation} below establishes implicit extrapolation, i.e.~that the solution to which GF converges extrapolates, \emph{irrespective of how large the student's state space dimension~$d$ is}.
A condition posed by the theorem is that the training sequence length~$k$ is greater than two times the teacher's state space dimension~\smash{$\hat{d}$}.
%This condition is required for the information entailed in training sequences to allow extrapolation.\amirg{I would drop this sentence, as it again points to a DOF direction. And rephrase the following sentence e..g, "In sec.. we show empirically smaller $k$ values result in failure to extrapolate."}
We empirically demonstrate the necessity of this condition in \Secref{sec:experiments}, by showing that smaller values for~$k$ lead to failure in extrapolation.
\begin{theorem}\label{thm:exact_extrapolation}
Assume that $d > k > 2\hat{d}$, the teacher parameters~$\hat{\Theta}$ are balanced (Definition~\ref{def:balanced}), and the student parameters~$\Theta$ are learned by applying GF to the loss~$\mathcal{L} ( \cdot )$ (\Eqref{eq:population_loss}) starting from a balanced initialization.
Then, if GF converges to a point~$\Theta^*$ satisfying $\mathcal{L} ( \Theta^* ) = 0$, this point extrapolates (Definition~\ref{def:extrapolation}).
\end{theorem}

In order to prove Theorem~\ref{thm:exact_extrapolation}, we introduce two lemmas:
Lemma~\ref{lemma:balanced}, which shows that balancedness is preserved under GF;
and 
Lemma~\ref{lemma:exact_extrapolation}, which (through a surprising connection to a moment problem from statistics) establishes that a balanced solution attaining zero loss necessarily extrapolates. With Lemmas \ref{lemma:balanced} and~\ref{lemma:exact_extrapolation} in place, the proof of Theorem~\ref{thm:exact_extrapolation} readily follows.

\begin{restatable}{lemma}{balanced}
\label{lemma:balanced}
Let $\Theta(\tau)$, with $\tau \geq 0$, be a curve brought forth by applying GF to the loss~$\mathcal{L} ( \cdot )$ starting from a balanced initialization.
Then, $\Theta(\tau)$ is balanced for every $\tau \geq 0$.
\end{restatable}
\begin{sproof}[Proof sketch (for complete proof see \Appref{sec:lemma_balanced_proof}).]
The result follows from the symmetric role of $\mB$ and~$\mC$ in the loss $\mathcal{L} ( \cdot )$.
\end{sproof}

\begin{lemma}\label{lemma:exact_extrapolation}
Suppose that $d > k > 2\hat{d}$, the teacher is balanced, and that the student parameters $\Theta$ are balanced and satisfy $\mathcal{L} ( \Theta ) = 0$. Then $\Theta$ extrapolates.
\end{lemma}
\begin{sproof}[Proof sketch (for complete proof see \Appref{sec:exact_extrapolation_proof}).]
The proof is based on a surprising connection to the \textit{moment problem} from statistics (recovery of a probability distribution from its moments), which has been studied for decades (see, e.g., \cite{schmudgen2017moment}). 

Without loss of generality, we may assume that \smash{$\hat{\mA}$} is diagonal (if this is not the case then we apply an orthogonal eigendecomposition to~\smash{$\hat{\mA}$} and subsequently absorb eigenvectors into \smash{$\hat{\mB}$} and~\smash{$\hat{\mC}$}).
We may also assume that \smash{$\hat{\mC}\hat{\mB} = 1$} (otherwise we absorb a scaling factor into $\mB$ and/or~$\mC$).
%Without loss of generality we assume $\hat{\mC}\hat{\mB}=1$.\footnote{We assume the teacher is balanced and therefore $\hat{\mC}\hat{\mB}=\sum_{i=1}^{\hat{d}}b_i^2$ is non-negative. Denote $\hat{\mC}\hat{\mB}\equiv\eta$ and suppose $\eta\neq 1$. The system defined by $\Theta'=(\frac{\hat{\mC}}{\sqrt{\eta}}, \hat{\mA}, \frac{\hat{\mB}}{\sqrt{\eta}})$ has the same impulse response as $\hat{\Theta}$ scaled by $\frac{1}{\eta}$ and as a result satisfies $\mC'\mB'=1$.} 
Since \smash{$\hat{\mC}^\top=\hat{\mB}$} (teacher is balanced), we may define a probability vector (i.e.~a vector with non-negative entries summing up to one) \smash{$\hat{\vp} \in \mathbb{R}^{\hat{d}}$} via \smash{$\hat{p}_i=\hat{\emC}_i\hat{\emB}_i$}, \smash{$i = 1 , \ldots , \hat{d}$}.
We let \smash{$\hat{Z}$} denote the random variable whose support is \smash{$\{ \hat{A}_{1 1} , \ldots , \hat{A}_{\hat{d} \hat{d}} \}$} and which assumes the value \smash{$\hat{A}_{i i}$} with probability~\smash{$\hat{p}_i$}, \smash{$i = 1 , \ldots , \hat{d}$}.
%From the balanced assumption we have $\hat{\mC}^\top=\hat{\mB}$ and therefore if we define 
%$\hat{p}_i=\hat{\emC}_i\hat{\emB}_i$ we have that $\hat{p}_i\geq 0$ and $\sum_i \hat{p}_i=1$ and therefore $\hat{\vp}$ may be interpreted as a distribution over a random variable with $\hat{d}$ possible values. We shall assume that these values are $\hat{\emA}_1,\ldots,\hat{\emA}_{\hat{d}}$, and denote the corresponding random variable by $Z$.
Notice that for every $j \in \mathbb{N}$:
\[
\hat{\mC} \hat{\mA}^j \hat{\mB} = \sum\nolimits_{i=1}^{\hat{d}} \hat{p}_i \hat{\emA}_{i i} ^j = \E[\hat{Z}^j]
\text{\,,}
\]
meaning that the elements of the teacher's impulse response are precisely the moments of~\smash{$\hat{Z}$}.
%Furthermore, we can also interpret elements of the impulse response of $\hat{\Theta}$ as moments of this distribution.
%Let us write the $n^{th}$ element of the impulse response as:
%\begin{equation}
%\hat{\mC} \hat{\mA}^n \hat{\mB} = \sum_i \hat{p}_i \hat{\emA}_i ^n = \E_{\hat{\vp}}[\hat{Z}^n],
%\end{equation}
%where $\E_{\vp}[Z]$ is the expected value of a random variable $Z$ under the distribution $p$.

Similarly to above we may assume~$\mA$ is diagonal, and since $\mathcal{L} ( \Theta ) = 0$ it holds that \smash{$\mC \mB = \hat{\mC}\hat{\mB} = 1$}.
We may thus define a probability vector $\vp \in \mathbb{R}^{d}$ via $p_i=\emC_i\emB_i$, $i = 1 , \ldots , d$, and a random variable~$Z$ which assumes the value $A_{i i}$ with probability~$p_i$, $i = 1 , \ldots , d$.
For every $j \in \mathbb{N}$:
\[
\mC \mA^j \mB = \sum\nolimits_{i=1}^d {p}_i \emA_{i i} ^j = \E[ Z^j]
\text{\,,}
\]
and so the elements of the student's impulse response are precisely the moments of~$Z$.
%Similarly, we can define for the learned model $\Theta$, a distribution $p_i = \emC_i \emB_i$, and write the learned impulse response as:
%\begin{equation}
%{\mC} {\mA}^n {\mB} = \sum_i {p}_i \emA_i ^n = \E_{\vp}[{Z}^n].
%\end{equation}

The probabilistic formulation we set forth admits an interpretation of extrapolation as a moment problem.
Namely, since $\mathcal{L} ( \Theta ) = 0$ (i.e.~\smash{$\mC \mA^j \mB=\hat{\mC} \hat{\mA}^j \hat{\mB}$} for $j=0,\dots,k-1$) the random variables $Z$ and~\smash{$\hat{Z}$} agree on their first $k - 1$ moments, and the question is whether they agree on all higher moments as well.
We note that this question is somewhat more challenging than that tackled in classic instances of the moment problem, since the support of the random variable whose moments we match~($Z$) is not known to coincide with the support of the random variable we seek to recover~(\smash{$\hat{Z}$}).
Luckily, a recent powerful result allow addressing the question we face~---~\cite{cohen2011use} showed that the first $2 n$ moments of a discrete random variable~$X$ taking at most $n \in \mathbb{N}$ values uniquely define~$X$, in the sense that \emph{any} discrete random variable agreeing with these $2 n$ moments must be identical to~$X$.
Translating this result to our setting, we have that if~$Z$ agrees with~\smash{$\hat{Z}$} on its first~\smash{$2 \hat{d}$} moments, it must be identical to~\smash{$\hat{Z}$}, and in particular it must agree with~\smash{$\hat{Z}$} on all higher moments as well. 
The fact that \smash{$k - 1 \geq 2 \hat{d}$} then concludes the proof.

%This view provides us with a moment matching interpretation of the learning problem. Namely, the fact that $\Theta$ matches the first $k$ elements of the teacher impulse response, is the same as saying they agree on the first $k-1$ moments $\E_{\vp}[{Z}^i]$ for $i\in\{1,\ldots, k-1\}$.\footnote{The matching of the $0^{th}$ moment ensures that the student also satisfies $\mC\mB=1$, and along with balancedness admits the random variable interpretation.} The question of extrapolation is whether equality in the first $k-1$ moments implies an equality in all other moments.
 %We note that there is a slight ``twist'' here on the classic moment problems because $\hat{Z}$ and $Z$ are also different so the random variables whose moments are compared are not the same in our setting.\amirg{(i.e., both their values and their probabilities differ)}

%Luckily, there are powerful results that address exactly the moment question above. In \citet[Theorem 1]{cohen2011use} it is shown that the first $2\hat{d}$ moments of a discrete random variable taking at most $\hat{d}$ different values uniquely define this random variable. Therefore, any other discrete random variable identifying with the teacher on $2\hat{d}$ moments must be the same random variable and therefore identifies on higher moments as well. Since we assumed $k > 2\hat{d}$, this result immediately implies that equality in the first $k-1$ moments implies equality in all other moments.

To attain some intuition for the result we imported from \cite{cohen2011use}, consider the simple case where \smash{$\hat{d}=1$}.
The transition matrix \smash{$\hat{\mA}$} is then a scalar \smash{$\hat{a} \in \mathbb{R}$}, the random variable \smash{$\hat{Z}$} is deterministically equal to~\smash{$\hat{a}$}, and the teacher's impulse response is given by the moments \smash{$\E[\hat{Z}^j] = \hat{a}^j$}, $j = 0 , 1 , \ldots$.
Since by assumption \smash{$k > 2 \hat{d}$}, the fact that $\mathcal{L} ( \Theta ) = 0$ means the random variable corresponding to the student, $Z$, agrees with the first two moments of~\smash{$\hat{Z}$}.
That is, $Z$~satisfies \smash{$\E[Z] = \hat{a}$} and \smash{$\E[Z^2] = \hat{a}^2$}.
This implies that $\Var[Z]=\E[Z^2] - \E[Z]^2 = 0$, and therefore $Z$ is deterministically equal to~\smash{$\hat{a}$}, i.e.~it is identical to~\smash{$\hat{Z}$}.
The two random variables thus agree on all of their moments, meaning the impulse responses of the student and teacher are the same.
%
%To provide some simple intuition for this result, consider the case of $\hat{d}=1$ (i.e. the teacher is governed by $\hat{a}\in\R$) and $k=3$. Then the teacher impulse response elements are $\E_{\hat{\vp}}[\hat{Z}^n] =\hat{a}^n$. If the student matches the first three of these (since $k=3$), it means it has found a random variable $Z$ and distribution $\vp$ with first and second moments $\E_{\vp}[Z] = \hat{a}_1,\E_{\vp}[Z^2]=\hat{a}_1^2$. But this implies that $\Var[Z]=\E_{\vp}[Z^2] - \E_{\vp}^2[Z] = 0$. As the variance of $Z$ is zero, $Z\equiv \hat{a}_1$. Thus, $\E_{\vp}[{Z}^n] = \hat{a}^n$ for all $n$ and the student matches the impulse response of the teacher, and therefore extrapolates.
\end{sproof}

% With Lemmas \ref{lemma:balanced} and~\ref{lemma:exact_extrapolation} in place, the proof of Theorem~\ref{thm:exact_extrapolation} readily follows.
\begin{proof}[Proof of Theorem~\ref{thm:exact_extrapolation}]
By Lemma~\ref{lemma:balanced} (as well as continuity considerations) $\Theta^*$~is balanced.
Lemma~\ref{lemma:exact_extrapolation} then implies that it extrapolates.
\end{proof}

\subsection{Approximate Convergence Leads to Approximate Extrapolation} \label{sec:approx_extrapolation}

Theorem~\ref{thm:exact_extrapolation} in \Secref{sec:exact_extrapolation} proves extrapolation in the case where GF converges to a zero loss solution.
Theorem~\ref{thm:main_result} below extends this result by establishing that, as long as the teacher is \emph{stable}, approximate convergence leads to approximate extrapolation.
\begin{theorem}\label{thm:main_result}
Assume the conditions of Theorem~\ref{thm:exact_extrapolation}, and that the teacher parameters~$\hat{\Theta}$ are stable, i.e. the singular values of~\smash{$\hat{\mA}$} are in~$[-1,1]$.
Assume also that~\smash{$\hat{\Theta}$} are non-degenerate, in the sense that the input-output mapping they realize is not identically zero.
Then, for any $\epsilon > 0$ and $q \in \mathbb{N}$, there exists $\delta(\epsilon, q) > 0$ which admits the following:
whenever GF leads the student parameters~$\Theta$ to within (Frobenius) distance $\delta(\epsilon,q)$ from a balanced point~$\Theta^*$ satisfying $\mathcal{L} ( \Theta^* ) = 0$, the student $\epsilon$-extrapolates with horizon~$q$. 
\end{theorem}
\begin{sproof}[Proof sketch (for complete proof see \Appref{sec:apdx:approx_extrapolation}).]
%\nadav{I did not review this}
In order to show $\epsilon$-extrapolation with horizon $q$ we need to show that $\forall\epsilon>0, q\in\mathbb{N}$, $\exists \delta(\epsilon,q)>0$ such that for any $\Theta'$ satisfying $||\Theta'-\Theta^*||<\delta$, we have for $n=0,\dots,q$ that
\begin{equation}
    \mC'(\mA')^n\mB'-\hat{\mC}\hat{\mA}^n\hat{\mB}=O(\epsilon).
\end{equation}

First, by Theorem~\ref{thm:exact_extrapolation} $\Theta^*$ is an extrapolating solution. Following the interpretation of discrete random variables of Lemma~\ref{lemma:exact_extrapolation} proof, we can bound the Wasserstein-1 distance between the extrapolating distribution defined by $\Theta^*$ and the approximating distribution defined by $\Theta'$. From Proposition 2 in \cite{wu2020optimal}, we have that if $|\mC'(\mA')^n\mB'-\mC^*(\mA^*)^n\mB^*|\le \delta$ for $n\in\lbrace 0,\dots,2\hat{d}\rbrace$, then
\begin{equation}
    \mathcal{W}_1(\Theta',\Theta^*)=O\left(\delta^{\frac{1}{4\hat{d}}}\right).
\end{equation}

The second part of the proof ties the error at step $p$ with the $\mathcal{W}_p$ distance \citep{biswas2021bounding}:
\begin{equation}
    \mC'(\mA')^p\mB'-\mC^*(\mA^*)^p\mB^*\le \mathcal{W}_p.
\end{equation}
The final step of the proof is based on \cite{panaretos2019statistical} and bounds $\mathcal{W}_p$ in terms of $\mathcal{W}_1$, i.e.,
\begin{equation}
    \mathcal{W}_p(\Theta,\Theta^*)= S^{p-1}\mathcal{W}_1(\Theta,\Theta^*),
\end{equation}
where $S$ is the difference between the maximal and minimal eigenvalues in the teacher and student support (due to our assumption on convergence, $S$ is bounded).


Combining the three steps above, we have that if $|\mC'(\mA')^n\mB'-\mC^*(\mA^*)^n\mB^*|\le \delta$ for $n\in\lbrace 0,\dots,2\hat{d}\rbrace$, then $\forall p>n$ \begin{equation}\label{eq:final_inequality}
    \mC'(\mA')^p\mB'-\mC^*(\mA^*)^p\mB^*\le S^{p-1}O\left(\delta^{\frac{1}{4\hat{d}}}\right).
\end{equation}

By setting $\delta<\left(\frac{\epsilon}{c_1\cdot S^{q-1}}\right)^{4\hat{d}}$ the RHS of \Eqref{eq:final_inequality} satisfies $c_1S^{p-1}\delta^{\frac{1}{4\hat{d}}}<\epsilon$.
% In order to have $\epsilon$-extrapolation up to horizon $q$, we can set $\delta<\left(\frac{\epsilon}{c_1\cdot S^{q-1}}\right)^{4\hat{d}}$ and then the RHS of \Eqref{eq:final_inequality} satisfies $c_1S^{p-1}\delta^{\frac{1}{4\hat{d}}}<\epsilon$.
\end{sproof}

\ignore{
\begin{corollary}\label{corollary:extrapolation_for_any_horizon}
\amirg{not sure this corollary is helpful. Also seems strange that thing will work in the non-stable case so need to check this carefully.}
If $\max_{i} A^*_i-\min_{j}A^*_j<1$ we have extrapolation for an infinite horizon.
\end{corollary}
The proof for Corollary \ref{corollary:extrapolation_for_any_horizon} follows by noting that $\exists\delta$ such that if $|\Theta'-\Theta^*|<\delta$, then $S<1$. With $S<1$ the final $\delta$ term does not depend on $q$ and therefore extrapolates for an infinite horizon.
}

\subsection{Balancedness Captures Near-Zero Initialization}
\label{sec:relaxed_balanced_assumption}

Theorems \ref{thm:exact_extrapolation} and~\ref{thm:main_result} assume that GF emanates from a balanced initialization, i.e.~from a point $\Theta = ( \mA , \mB , \mC )$ satisfying $\mB = \mC^\top$.
It was shown in~\cite{cohen2022extrapolation} that theoretical predictions derived assuming balanced initialization faithfully match experiments conducted with near-zero initialization (an initialization commonly used in practice).
Proposition~\ref{prop:implicit_bias_for_balanced_init} below theoretically supports this finding, establishing that with high probability, random near-zero initialization leads GF to arrive at an approximately balanced point, i.e.~a point $\Theta = ( \mA , \mB , \mC )$ for which the difference between $\mB$ and~$\mC^\top$ is negligible compared to their size.
\begin{proposition}\label{prop:implicit_bias_for_balanced_init}
Suppose that:
\emph{(i)}~$d > 4$;
\emph{(ii)}~the teacher parameters~$\hat{\Theta}$ are balanced and are non-degenerate, in the sense that the input-output mapping they realize is not identically zero;
and 
\emph{(iii)}~the student parameters are learned by applying GF to the loss~$\mathcal{L} ( \cdot )$.
Let $\tilde{\Theta}$ be a random point in parameter space, with entries drawn independently from the standard normal distribution.
For $\epsilon > 0$, consider the case where GF emanates from the initialization~$\epsilon \tilde{\Theta}$, and denote the resulting curve by $\Theta_\epsilon ( \tau ) = ( \mA_\epsilon ( \tau ) , \mB_\epsilon ( \tau ) , \mC_\epsilon ( \tau ) )$, with $\tau \geq 0$.
Then, w.p. at least~$0.75$, for every $\epsilon > 0$ there exists $\tau_\epsilon \geq 0$ such that:
\begin{equation}
    \lim_{\epsilon\rightarrow 0^+}\frac{||\mB_{\epsilon}(\tau_\epsilon) - \mC_{\epsilon}^\top(\tau_\epsilon) ||_F}{||\mB_{\epsilon}(\tau_\epsilon) + \mC_{\epsilon}^\top(\tau_\epsilon) ||_F}=0
    \text{\,.}
\end{equation}
\end{proposition}
\begin{sproof}[Proof sketch (for complete proof see \Appref{sec:apdx:bias_to_symmetry}).]
The idea behind the proof is as follows.
Assume $\epsilon$ is sufficiently small.
Then, when the entries of $\Theta = ( \mA , \mB , \mC)$ are on the order of~$\epsilon$, we have \smash{$\frac{\partial}{\partial \mB} \mathcal{L} ( \Theta ) \approx -2 \hat{\mC} \hat{\mB} \cdot \mC^\top$} and \smash{$\frac{\partial}{\partial \mC} \mathcal{L} ( \Theta ) \approx -2 \hat{\mC} \hat{\mB} \cdot \mB^\top$}.
This implies that during the first part of the curve~$\Theta_\epsilon ( \tau )$ it holds that
$\tfrac{d}{d \tau} ( \mB_\epsilon ( \tau ) - \mC_\epsilon^\top ( \tau ) ) \approx - 2 \hat{\mC} \hat{\mB} \cdot ( \mB_\epsilon ( \tau ) - \mC_\epsilon^\top ( \tau ) )$
and similarly
$\tfrac{d}{d \tau} ( \mB_\epsilon ( \tau ) + \mC_\epsilon^\top ( \tau ) ) \approx ~~~ 2 \hat{\mC} \hat{\mB} \cdot ( \mB_\epsilon ( \tau ) + \mC_\epsilon^\top ( \tau ) )$.
%\begin{eqnarray*}
%\tfrac{d}{d \tau} ( \mB_\epsilon ( \tau ) - \mC_\epsilon^\top ( \tau ) ) &\approx& - 2 \hat{\mC} \hat{\mB} \cdot ( \mB_\epsilon ( \tau ) - \mC_\epsilon^\top ( \tau ) )
%\text{\,,} \\
%\tfrac{d}{d \tau} ( \mB_\epsilon ( \tau ) + \mC_\epsilon^\top ( \tau ) ) &\approx& ~~~ 2 \hat{\mC} \hat{\mB} \cdot ( \mB_\epsilon ( \tau ) + \mC_\epsilon^\top ( \tau ) )
%\text{\,.}
%\end{eqnarray*}
Since $\hat{\mC} \hat{\mB} > 0$ (follows from the teacher parameters being balanced and non-degenerate), the entries of $\mB_\epsilon ( \tau ) - \mC_\epsilon^\top ( \tau )$ shrink exponentially fast while those of $\mB_\epsilon ( \tau ) + \mC_\epsilon^\top ( \tau )$ grow at the same rate.
This exponential shrinkage/growth leads $\| \mB_\epsilon ( \tau ) - \mC_\epsilon^\top ( \tau ) \| \big/ \| \mB_\epsilon ( \tau ) + \mC_\epsilon^\top ( \tau ) \|$ to become extremely small, more so the smaller $\epsilon$ is.
%
% OLD PROOF SKETCH:
%The proof involves bounding the differential equations characterizing $\mC^\top + \mB$ and $\mC^\top - \mB$, and showing that their dynamics at initialization are governed by the dominating terms of the gradients. Both terms turn out to be exponential, with the term corresponding to $\mC^\top + \mB$ having an exponent of the same sign as $\hat{\mC}\hat{\mB}$, and the term corresponding to $\mC^\top - \mB$ having an exponent of the opposite sign.
%s of opposite signs, and include the term  which determines which term will decay and which will grow. 
%We then show that there exists a time for which the two terms are separated by an order of magnitude. Finally, taking $\epsilon\rightarrow 0$, we have that the limit is unbounded.
\end{sproof}